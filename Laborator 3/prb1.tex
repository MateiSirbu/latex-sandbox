\documentclass[9pt,a4paper]{article}
\usepackage{amsfonts,amsmath,amsthm,indentfirst,fancyvrb,alltt}
\usepackage[utf8x]{inputenc}
\usepackage[table]{xcolor}
\renewcommand\refname{Bibliografie}

\title{\bf Laborator 3}
\author{Sîrbu Matei-Dan}
\date{29 octombrie 2020}

\begin{document}
\maketitle

\section{Texturi și imagini color}

Domeniul ,,nativ'' al aplicațiilor CUDA este procesarea imaginilor.\cite{andonie} Acest domeniu depășește cadrul laboratorului; ne vom rezuma doar la câteva example cu rol introductiv. Scopul acestei lucrări este de a prezenta câteva exemple mai complexe de programe CUDA, exemplificând în mod special procesarea cu texturi și imagini.

\subsection{Procesarea texturilor}
Un termen împrumutat din grafica 3D -- textura -- poate fi considerată o matrice uni-, bi- sau tri-dimensională de \textit{texeli} (texture-elements). Texelii pot fi reprezentați prin scalari (byte, float), sau 4-tuple (byte4, float4).

În CUDA, texturile se disting ca o zonă de memorie specială, care poate fi citită cu ajutorul unor funcții de acces speciale \texttt{tex1D(x), tex2D(x, y)}, respectiv \texttt{tex3D(x, y, z)}. Texturile oferă următoarele facilități:

\begin{itemize}
    \item pentru dispozitivele mai vechi, citirea din memoria de textură este mai  rapidă decât accesul din memoria globală dispozitiv\footnote{Arhitectura CUDA 2.x (Fermi) oferă memorie cache și memoriei globale},
    \item se pot citi și elemente de la coordonate ne-întregi, interpolarea (lineară) a valorilor efectuându-se de către hardware (de ex: \texttt{float a = tex2D(1.5, 3.25) }),
    \item coordonatele care depășesc domeniul texturii $[0\dots N-1]$ se ajustează automat, fie forțându-șe la marginile domeniilor $0, N-1$ fie calculând modulo $N$, configurabil, 
\end{itemize}

\newpage

\section*{Primul program CUDA}

Scopul acestui exemplu este de a ne familiariza cu mediul CUDA, compilarea unui program, lucrul cu memoria dispozitiv, modelul de execuție multi-threaded divizat în blocuri.

În loc de tradiționalul \textit{"Hello World"}, primul exemplu va consta din stocarea în memorie a șirului de întregi 0, 1, 2, \dots\  N-1, echivalentul paralel al următorului cod C:
\begin{Verbatim}[xleftmargin=.5in]
for(i=0; i<N; i++)
        a[i]=i;
\end{Verbatim}
Chiar și pentru acest banal exemplu, este nevoie de următorii pași:
\begin{enumerate}
    \item alocarea memoriei pe dispozitiv și gazdă (pointerii a\_cpu, respectiv a\_gpu)
    \item apelul CUDA: Lansarea a $N$ threaduri. Fiecare thread are asociat un indice unic, intrinsec, de la $0$ la $N-1$. Pseudo-codul este:
\begin{Verbatim}
Thread(i) executa:
    a[i]=i;
terminare
\end{Verbatim}
\item așteptarea terminării tuturor threadurilor
\item copierea rezultatului din memoria dispizitiv în memoria gazdă\footnote{Copierea are loc, de obicei pe magistrala pe care se află placa video, iar viteza acestuia limitând performanța sistemului, se caută minimizarea numărului de transferuri.}
\item afișarea șirului rezultat.
\end{enumerate}
În pașii 2-3, execuția se desfășoară pe procesorul grafic, iar procesorul gazdă așteaptă rezultatele în procedura \texttt{cudaThreadSynchronize()}. În continuare vom prezenta codul unei posibile implementări CUDA, folosind un vector de threaduri.
% === VARIANTA FĂRĂ ALLTT ===
% \begin{Verbatim}[commandchars=\\~&]
% \textit~// Codul care va rula pe GPU&
% __global__ void threadScriere(int * a, int n) {
%         \textit~// indicele threadului&
%         int i = blockDim.x * blockIdx.x + threadIdx.x;
%         if (i < n)
%                 a[i] = i;
% }
% \textit~// functia main() din C standard, se executa pe CPU&
% int main(int argc, char ** argv) {
%     int n = 1024; \textit~// dimensiune&
%     int *a_cpu;   \textit~// pointer la memoria gazda&
% }
% \end{Verbatim}
% === VARIANTA CU ALLTT ===
\begin{alltt}
{\it \texttt{// Codul care va rula pe GPU}}
__global__ void threadScriere(int * a, int n) {\{}
        {\it \texttt{// indicele threadului}}
        int i = blockDim.x * blockIdx.x + threadIdx.x;
        if (i < n)
                a[i] = i;
{\}}
{\it \texttt{// functia main() din C standard, se executa pe CPU}}
int main(int argc, char ** argv) {\{}
    int n = 1024; {\it \texttt{// dimensiune}}
    int *a_cpu;   {\it \texttt{// pointer la memoria gazda}}
\end{alltt}

Etapele unui program care folosește CUDA sunt, de obicei:
\begin{enumerate}
    \item alocarea memoriei pe dispozitiv,
    \item copierea datelor de intrare din memoria gazdă în memoria dispozitiv,
    \item apel nucleu CUDA: lansarea unei grile de threaduri,
    \item calcul pe dispozitivul GPU
    \item sincronizarea gazdă-dispozitiv prin așteptarea terminării execuției tuturor threadurilor,
    \item preluarea rezultatului din memoria dispozitiv.
\end{enumerate}

Pașii 1-3,5-6 sunt executate de către calculatorul gazdă, iar 4 este executat de  procesorul grafic. În acest timp, procesorul gazdă (CPU) este inactiv, sau poate fi programat pentru alte activități. Astfel, putem considera GPU ca un \textit{ co-procesor} care poate prelua secțiunile de calcul intensiv din program.

\begin{thebibliography}{99}
    \bibitem{andonie}Răzvan Andonie, Angel Cațaron, Honorius Gâlmeanu, Zoltan Gaspar, Lorentz Istvan, Mihai Ivanovici, Lucian Sasu, \textit{Algoritmi și structuri de date, Îndrumar de laborator}, Universitatea ,,Transilvania'' din Brașov, 2012
\end{thebibliography}

\end{document}