\documentclass[9pt,a4paper]{article}
\usepackage{amsfonts,amsmath,amsthm,indentfirst}
\usepackage[utf8x]{inputenc}

\title{\bf Exemple de tabele}
\author{Sîrbu Matei-Dan}
\date{15 octombrie 2020}

\begin{document}
\maketitle

\section*{Tabelul 1}

\begin{verbatim}
    \begin{tabular}{|r|l|c|}
        \hline\hline
        Nr.crt   & Nume    & Ocupația \\
        \hline\hline
        1.       & Popescu & Inginer  \\
        \hline2. & Ionescu & Profesor \\
        \hline3. & Andrei  & Șomer    \\
        \hline
    \end{tabular}
\end{verbatim}

\begin{tabular}{|r|l|c|}
    \hline\hline
    Nr.crt   & Nume    & Ocupația \\
    \hline\hline
    1.       & Popescu & Inginer  \\
    \hline2. & Ionescu & Profesor \\
    \hline3. & Andrei  & Șomer    \\
    \hline
\end{tabular}

\section*{Tabelul 2}

\begin{verbatim}
    \begin{tabular}{|l l c|}
        \hline
        Nr.crt&Nume& Ocupația\\
        1. & Popescu & Inginer \\
        2. & Ionescu & Profesor \\
        3. & Andrei & Șomer \\
        \hline
    \end{tabular}
\end{verbatim}

\begin{tabular}{|l l c|}
    \hline
    Nr.crt & Nume    & Ocupația \\
    1.     & Popescu & Inginer  \\
    2.     & Ionescu & Profesor \\
    3.     & Andrei  & Șomer    \\
    \hline
\end{tabular}

\pagebreak

\section*{Tabelul 3}

\begin{verbatim}
    \begin{tabular}{|p{4cm}|}
        \hline
        Acesta este exemplu
        de paragraf scris
        \^\i n cutie. \\
        \hline
    \end{tabular}
\end{verbatim}

\begin{tabular}{|p{4cm}|}
    \hline
    Acesta este exemplu
    de paragraf scris
    \^\i n cutie. \\
    \hline
\end{tabular}

\section*{Tabelul 4}

\begin{verbatim}
    \hline\hline
    \multicolumn{2}{|c|}{Denumirea} & Cant.   &
    \multicolumn{2}{c|}{Preț (mii lei)}                              \\
    \cline{4-5}
    \multicolumn{2}{|c|}{}          &         & unitar & total       \\
    \hline\hline
    \multicolumn{2}{|c|}{Roșii}     & 3 Kg    & 0,3    & 0,9         \\
    \hline
    Carne                           & Cal. I  & 2 Kg   & 4     & 8   \\
    \cline{2-5}
                                    & Cal. II & 3 Kg   & 2,5   & 7,5 \\
    \hline\hline
\end{verbatim}

\begin{tabular}{|c|c|c|r|r|}
    \hline\hline
    \multicolumn{2}{|c|}{Denumirea} & Cant.   &
    \multicolumn{2}{c|}{Preț (mii lei)}                              \\
    \cline{4-5}
    \multicolumn{2}{|c|}{}          &         & unitar & total       \\
    \hline\hline
    \multicolumn{2}{|c|}{Roșii}     & 3 Kg    & 0,3    & 0,9         \\
    \hline
    Carne                           & Cal. I  & 2 Kg   & 4     & 8   \\
    \cline{2-5}
                                    & Cal. II & 3 Kg   & 2,5   & 7,5 \\
    \hline\hline
\end{tabular}

\end{document}