\documentclass[9pt,a4paper]{article}
\usepackage{amsfonts,amsmath,amsthm}
\usepackage[utf8x]{inputenc}
\newtheorem{teorema}{Teorema}
\newtheorem{prop}{Propoziția}
\renewcommand{\proofname}{Demonstrație}

\title{\bf Primul meu document în \LaTeX{}}

\author{Sîrbu Matei-Dan}

\date{15 octombrie 2020}

\begin{document}
\maketitle

\begin{abstract}
    Acesta este primul meu document scris în \LaTeX{}.
\end{abstract}

\section{Introducere}
    În continuare vom argumenta de ce \LaTeX{} este indicat pentru redactarea textelor și a formulelor matematice.

\begin{itemize}
    \item este un program stabil pe diverse platforme;
    \item aduce noi îmbunătățiri în ce privește calitatea și ușurința de redactare, cât și o afișare profesională.
\end{itemize}

\section{Rezultate utilizate în document}

Acest document folosește următoarele rezultate teoretice:
\begin{teorema}\label{pitagora}
    Dacă în triunghiul $\triangle ABC$ unghiul $\hat A$ este un unghi drept, atunci $$BC^2=AB^2+AC^2.$$
\end{teorema}

\begin{prop}
    Teorema \ref{pitagora} este valabilă doar în spații euclidiene.
\end{prop}

Textul matematic se introduce prin inserarea de dolar în ambele capete astfel:
$\frac{\pi}{4}=\sum_{k-1}^{\infty}\frac{(-1)^{k+1}}{2k-1}$.

Dacă se dorește afișarea ecuației pe următorul rând centrată se folosește dublu dolar în ambele capete astfel:

$$\frac{\pi}{4}=\sum_{k=1}^{\infty}\frac{(-1)^{k+1}}{2k-1}$$

Dacă dorim să numerotăm ecuația folosim mediul:

\begin{equation}\label{eq1}
    g(x)=\int_{0}^{\pi}\sin(x)dx
\end{equation}

\section{Concluzii}

Cu ajutorul unui număr mic de comenzi, ușor de înțeles putem realiza un document având o calitate tipografică deosebită.

\end{document}