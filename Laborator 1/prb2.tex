\documentclass[9pt,a4paper]{article}
\usepackage{amsfonts,amsmath,amsthm,indentfirst}
\usepackage[utf8x]{inputenc}
\renewcommand{\labelitemi}{\textbullet}
\renewcommand{\labelitemii}{\textbullet}

\begin{document}

Ca și în cazul algoritmilor greedy, soluția optimă nu este în mod necesar unică.
Dezvoltarea unui algoritm de programare dinamică poate fi descrisă de următoarea succesiune de pași:

\begin{itemize}
    \item Se caracterizează structura unei soluții optime
    \item Se definește recursiv valoarea unei soluții optime
    \item Se calculează de jos în sus valoarea unei soluții optime
    \newline Dacă pe lângă valoarea unei soluții optime se dorește și soluția 
    propriu-zisă atunci se mai efectuează și acest pas:
    \begin{itemize}
        \item Din informațiile calculate se construiește de sus în jos o soluție optimă.
    \end{itemize}
\end{itemize}

Să ne imaginăm o competiție în care doi jucători $A, B$ joacă o serie de cel mult $2n - 1$ partide, câștigător
fiind jucătorul care acumulează primul $n$ victorii. Presupunem că nu există partide egale, și că rezultatele sunt
independente între ele și că pentru orice partidă există o probabilitate $p$ ca jucătorul $A$ să câștige,
și o probabilitate $1 - p$ ca jucătorul B să câștige.

Ne propunem să calculăm $P(i,j)$, probabilitatea ca jucătorul $A$ să câștige competiția, dat fiind că mai are nevoie
de $i$ victorii, iar jucătorul $B$ mai are nevoie de $j$ victorii pentru a câștiga. La început evident, probabilitatea
este $P(n,n)$ pentru că fiecare jucător mai are nevoie de $n$ victorii.

Pentru $1 \leq i \leq n$ avem $P(0, i) = 1$ implică $P(i, 0) = 0$. Probabilitatea $P(0,0)$ este nedefinită.

Pentru $i, j \geq 1$ se poate calcula $P(i, j)$ după formula:
$$P(i,j) = pP(i-1,j) + qP(i,j-1)$$

\end{document}