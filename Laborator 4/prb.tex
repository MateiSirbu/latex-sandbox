\documentclass[9pt,a4paper]{article}
\usepackage{amsfonts,amsmath,amsthm,xpatch}
\usepackage[utf8x]{inputenc}
\renewcommand\refname{Bibliografie}
\newtheorem*{definitie}{Definiția}
\makeatletter
\xpatchcmd{\@thm}{\thm@headpunct{.}}{\thm@headpunct{:}}{}{}
\makeatother

\title{\bf Laborator 4}
\author{Sîrbu Matei-Dan}
\date{5 noiembrie 2020}

\begin{document}
\maketitle

\section*{Repartiția binomială $\mathbf{B}_i(\mathbf{n;p})$}

\begin{definitie}O variabilă aleatoare X are o repartiție binomială de parametrii n și p, dacă repartiția sa are forma:
\end{definitie}
$$X\begin{pmatrix}
    k \\ C_n^k p^k q^{n-k} 
\end{pmatrix}_{\substack{k=\overline{0,n} \\ n \in \mathbb{N}}}, p \geq 0, p + q = 1$$
Avem
    $$p_{n, k} = C_n^k p^k q^{n-k} \geq 0, k = \overline{0, n} \text{ și } \sum_{k=0}^n {p_{n,k}} = \sum_{k=0}^n{C_n^k p^k q^{n-k}} = (p + q)^n = 1.$$ 

\textbf{\textit{Funcția de repartiție}} a variabilei aleatoare X binomiale este:
    $$F(x) = P(X < x) = \sum_{k=0}^{[x]-1}{p_{n,k},\ [x] \text{ fiind partea întreagă a lui X.}}$$

\textbf{\textit{Funcția caracteristică}} corespunzătoare repartiției binomiale este:
\begin{align*}
&\varphi(t) = M(e^{itx}) = \sum_{k=0}^n{C_n^k e^{itk} p^k q^{n-k}} = \sum_{k=0}^n{C_n^k (e^{it} \cdot p)^k q^{n-k}} \\ 
&\varphi(t) = (pe^{it} + q)^n.
\end{align*}

\textbf{\textit{Media}} variabilei aleatoare $X$ binomiale este:
$$M(X) = np.$$

\textbf{\textit{Dispersia}} variabilei aleatoare $X$ binomiale este:
$$D^2(X) = npq.$$

\textbf{\textit{Abaterea pătratică medie}} a variabilei aleatoare $X$ binomiale este:

$$\sigma(X) = \sqrt{D^2(X)} = \sqrt{npq}.$$

\newpage

\section*{Repartiția uniformă continuă}
\begin{definitie}
O variabilă aleatoare X are o repartiție uniformă continuă de parametrii a și b, dacă densitatea sa de repartiție este:
\end{definitie}
$$\rho(x) = 
\begin{cases}
    \dfrac{1}{b - a}, &x \in [a, b], 0 \leq a \leq b \\ 0, &x \notin [a, b]
\end{cases}.$$

Avem $$\int_{-\infty}^{\infty} \rho (x) \mathop{dx} = 1.$$

\textbf{\textit{Media}} variabilei aleatoare $X$ cu repartiție uniformă de parametrii $a$ și $b$ este:

$$M(x) = \frac{a+b}{2}.$$

\textbf{\textit{Dispersia}} variabilei aleatoare $X$ cu repartiția uniformă de parametrii a și b este:

$$D^2(x) = \frac{(a-b)^2}{12}.$$

\textbf{\textit{Funcția de repartiție}} a variabilei aleatoare $X$ cu repartiția uniformă de parametrii $a$ și $b$ este:

$$F(x) = \int_{-\infty}^x \rho(t) \mathop{dt} = 
\begin{cases}
    0, &x \leq a \\ \dfrac{x-a}{b-a}, &a \leq x \leq b \\ 1, &x > b
\end{cases}.$$

\textbf{\textit{Funcția caracteristică}} a variabilei aleatoare $X$ cu repartiția uniformă de parametrii $a$ și $b$ este:

$$\varphi(t) = 
\begin{cases}
    \dfrac{1}{it(b-a)}(e^{itb} - e^{ita}), &t \neq 0 \\ 1, &t = 0
\end{cases}.$$

\newpage

\section*{Sisteme de ecuații liniare}

Un ansamblu de egalități de forma:

\begin{equation} \label{eq:sistem}
    \begin{cases}
        a_{11}x_{1} + a_{12}x_2 + \dots + a_{1n}x_{n} = b_1 \\
        a_{21}x_{1} + a_{22}x_2 + \dots + a_{2n}x_{n} = b_2 \\
        \dotfill \\
        a_{m1}x_{1} + a_{m2}x_2 + \dots + a_{mn}x_{n} = b_m
    \end{cases}
\end{equation}
se numește \textit{sistem de m ecuații liniare cu n necunoscute} $x_1, x_2, \dots, x_n$.

Elementele $a_{ij}, i = \overline{1, m}, j = \overline{1, n}, a_{ij} \in \mathbb{R}$ se numesc \textit{coeficienți}, iar elementele $b_i \in \mathbb{R}, i = \overline{1,m}$ se numesc \textit{termeni liberi}.

Sistemul \eqref{eq:sistem} poate fi scris și sub formă matriceală:

$$A \cdot X = B \text{ sau } \sum_{j=1}^n a_{ij}x_j = b_i, i = \overline{1, m}$$

$$A = 
\begin{pmatrix}
    a_{11} & \dots & a_{1n} \\ 
    \dots & \dots & \dots \\ 
    a_{m1} & \dots & a_{mn}
\end{pmatrix}, 
B = \begin{pmatrix}
    b_1 \\ b_2 \\ \dots \\ b_m
\end{pmatrix}, 
X = \begin{pmatrix}
    x_1 \\ x_2 \\ \dots \\ x_n
\end{pmatrix}
$$

Matricea extinsă a sistemului \eqref{eq:sistem} este:

$$ \bar{A} = 
\begin{pmatrix}
    a_{11} & \dots & a_{1n} & b_1 \\ 
    \dots & \dots & \dots & \dots \\ 
    a_{m1} & \dots & a_{mn} & b_m
\end{pmatrix}
$$

\end{document}